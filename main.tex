\documentclass[aspectratio=169,t]{beamer}
\usepackage[utf8]{inputenc}

\usetheme{InfUfg}
%--------------------------------------- Primmary Definitions
% \setPrimmaryColor{UFGBlue} % This command set the default Color, is possible to choose a custom color

\setPrimmaryColor{INFBlue} 
% \setPrimmaryColor{DarkOrange}

\setLogos{lib/logos/infw.png}{lib/logos/infw2.png} %First one is logo in title slide (we recommend use a horizontal image), and second one is the logo used in the remaining slides (we recommend use a square image)


%----------------------------------------------------------PACKAGES -
\newtheorem{conj}{Conjetura}
\newtheorem{defi}{Definição}
\newtheorem{teo}{Teorema}
\newtheorem{lema}{Lema}
\newtheorem{prop}{Proposição}
\newtheorem{cor}{Corolário}
\newtheorem{ex}{Exemplo}
\newtheorem{exer}{Exercício}

% --------------------------------------Title Slide Information
\begin{document}
\title[Inf UFG]{Apresentações Instituto de informática}
\subtitle{Template Latex}

\author{Altino Dantas\inst{1} \and Deuslirio Junior\inst{2}}

\institute[UFG] % (optional)
{
  \inst{1}%
  Instituto de Informática\\
  Federal University of Goiás
  \and
  \inst{2}%
  Instituto de Informática\\
  Federal University of Goiás
}
\date{2019}
%-----------------------The next statement creates the title page.
\frame[noframenumbering]{\titlepage}



%------------------------------------------------Slide1
\setLayout{horizontal} %This command define the layout. 'horizontal' can be replace with 'vertical', 'blank, 'mainpoint', 'titlepage'

\begin{frame}
\frametitle{Table of Contents}
\tableofcontents
\end{frame}
%---------------------------------------------------------

%---------------------------------------------------------Slide2
\section{First section}

 \begin{frame}{Problemas mais complexos}
\footnotesize
  \begin{ex}
Em uma versão da linguagem BASIC, o nome de uma variável é uma sequência de um ou dois caracteres alfanuméricos, em que letras maiúsculas e minúsculas não são distinguidas. Além disso, um nome de variável deve começar com uma letra e deve ser diferente das cinco sequências de dois caracteres reservadas para o uso de comandos. Quantos nomes diferentes de variáveis são possíveis nesta versão do BASIC?
  \end{ex}
%\pause

\begin{block}{Solução}
Considere $V$ o número de nomes possíveis de variáveis diferentes do BASIC. Seja $V_1$ a quantidade de varíaveis com um caractere e $V_2$ a quantidade de variáveis com dois caracteres. Pela regra da soma, $V=V_1+V_2$. Como as variáveis só podem começar com letras, temos que $V_1=26$. Pela regra do produto, há $26\cdot 36=936$ sequências de tamanho $2$ que comecem com uma letra e terminam com um caracter alfanumérico. Porém, não se deve usar $5$ variáveis reservadas.
  \end{block}

 \end{frame}

%---------------------------------------------------------


%-------------------------------------------------------Slide3
\setLayout{vertical}
 \begin{frame}{Problemas mais complexos}
\footnotesize
  \begin{ex}
Em uma versão da linguagem BASIC, o nome de uma variável é uma sequência de um ou dois caracteres alfanuméricos, em que letras maiúsculas e minúsculas não são distinguidas. Além disso, um nome de variável deve começar com uma letra e deve ser diferente das cinco sequências de dois caracteres reservadas para o uso de comandos. Quantos nomes diferentes de variáveis são possíveis nesta versão do BASIC?
  \end{ex}
%\pause

\begin{block}{Solução}
Considere $V$ o número de nomes possíveis de variáveis diferentes do BASIC. Seja $V_1$ a quantidade de varíaveis com um caractere e $V_2$ a quantidade de variáveis com dois caracteres. Pela regra da soma, $V=V_1+V_2$. Como as variáveis só podem começar com letras, temos que $V_1=26$. Pela regra do produto, há $26\cdot 36=936$ sequências de tamanho $2$ que comecem com uma letra e terminam com um caracter alfanumérico. Porém, não se deve usar $5$ variáveis reservadas. Assim, $V_2=26\cdot 36-5=931$. Logo, há $V=V_1+V_2 = 26+931=957$ nomes diferentes para variáveis nesta versão do BASIC.
  \end{block}

 \end{frame}

%---------------------------------------------------------

%--------------------------------------------------------- Slide4
%Example of the \pause command

\begin{frame}
\frametitle{Pause Example}
In this slide \pause

the text will be partially visible \pause

And finally everything will be there
\end{frame}

%---------------------------------------------------------


%--------------------------------------------------------- Slide5
%Two columns
\section{Second section}

\setLayout{blank}
\begin{frame}
\frametitle{Two-column slide}

\begin{columns}

\column{0.5\textwidth}
This is a text in first column.
$$E=mc^2$$
\begin{itemize}
\item First item
\item Second item
\end{itemize}

\column{0.5\textwidth}
This text will be in the second column
and on a second tought this is a nice looking
layout in some cases.
\begin{enumerate}
    \item First
    \item Second
\end{enumerate}
\end{columns}
\end{frame}
%---------------------------------------------------------


%---------------------------------------------------------Slide6
%Highlighting text
\setLayout{vertical}
\begin{frame}
\frametitle{Sample frame title}

In this slide, some important text will be
\alert{highlighted} because it's important.
Please, don't abuse it.

\begin{block}{Remark}
Sample text
\end{block}

\begin{alertblock}{Important theorem}
Sample text in red box
\end{alertblock}

\begin{examples}
Sample text in green box. The title of the block is ``Examples".
\end{examples}
\end{frame}
%---------------------------------------------------------




%---------------------------------------------------------Slide7
\section{Preliminary Empirical Study}
\setLayout{mainpoint}

\begin{frame}{Preliminary Empirical Study}
\end{frame}

%-------------------------------------------------------

%---------------------------------------------------------Slide8
\setLayout{vertical}
 \begin{frame}{Problemas mais complexos}
\footnotesize
  \begin{ex}
Em uma versão da linguagem BASIC, o nome de uma variável é uma sequência de um ou dois caracteres alfanuméricos, em que letras maiúsculas e minúsculas não são distinguidas. Além disso, um nome de variável deve começar com uma letra e deve ser diferente das cinco sequências de dois caracteres reservadas para o uso de comandos. Quantos nomes diferentes de variáveis são possíveis nesta versão do BASIC?
  \end{ex}
%\pause

\begin{block}{Solução}
Considere $V$ o número de nomes possíveis de variáveis diferentes do BASIC. Seja $V_1$ a quantidade de varíaveis com um caractere e $V_2$ a quantidade de variáveis com dois caracteres. Pela regra da soma, $V=V_1+V_2$. Como as variáveis só podem começar com letras, temos que $V_1=26$. Pela regra do produto, há $26\cdot 36=936$ sequências de tamanho $2$ que comecem com uma letra e terminam com um caracter alfanumérico. Porém, não se deve usar $5$ variáveis reservadas.
  \end{block}

 \end{frame}

%---------------------------------------------------------

%---------------------------------------------------------Slide9
\subsection{Results}
% Example of changing background color 
\setLayout{vertical}
\setBGColor{DarkOrange}
\begin{frame}
\frametitle{Sample frame title}
This is a text in second frame. For the sake of showing an example.

\begin{itemize}
    \item<1-> Text visible on slide 1
    \item<2-> Text visible on slide 2
    \begin{itemize}
        \item text subitem
    \end{itemize}
    \item<3> Text visible on slides 3
    \item<4-> Text visible on slide 4
\end{itemize}
\end{frame}
%---------------------------------------------------------


%---------------------------------
\setLayout{titlepage}
\setBGColor{DarkGray}
\titlepage
%-------------------------------------
\end{document}