\documentclass[aspectratio=169]{beamer}
\usepackage[utf8]{inputenc}
% \usepackage{graphics}
% \usepackage{graphicx}
% \usepackage{booktabs}
% \usepackage{ragged2e}
% \usepackage{lipsum}
% \usepackage{minted}
% \usepackage{array}
% \usepackage{algorithm,algorithmicx}
% \usepackage{algpseudocode}
% \usepackage{amsmath,amsfonts,amssymb}
% \usepackage[export]{adjustbox}
% \usepackage{xcolor}

\usetheme{InfUfg}

\setPrimmaryColor{UFGBlue}
% \setPrimmaryColor{INFBlue} 
% \setPrimmaryColor{DarkOrange}



%---------------------------------------------------------------- PACKAGES -
\newtheorem{conj}{Conjetura}
\newtheorem{defi}{Definição}
\newtheorem{teo}{Teorema}
\newtheorem{lema}{Lema}
\newtheorem{prop}{Proposição}
\newtheorem{cor}{Corolário}
\newtheorem{ex}{Exemplo}
\newtheorem{exer}{Exercício}




% \makeatletter
\long\def\beamer@newenvnoopt#1#2#3#4{%
  \expandafter\renewcommand\expandafter<\expandafter>\csname#1\endcsname[#2]{#3}%<- here
  \expandafter\long\expandafter\def\csname end#1\endcsname{#4}%
}
\long\def\beamer@newenvopt#1#2[#3]#4#5{%
  \expandafter\renewcommand\expandafter<\expandafter>\csname#1\endcsname[#2][#3]{#4}%<- here
  \expandafter\long\expandafter\def\csname end#1\endcsname{#5}%
}
\makeatother

\renewenvironment<>{block}[1]{%
    \begin{actionenv}#2%
      \def\insertblocktitle{#1}%
      \par%
      \mode<presentation>{%
        \setbeamercolor{local structure}{use=block title,%
           fg=block title.bg}}%
      \usebeamertemplate{block begin}}
    {\par%
      \usebeamertemplate{block end}%
    \end{actionenv}}



\begin{document}
\title[Inf UFG] %optional
{Apresentações Instituto de informática}
\subtitle{Template Latex}




\author{Altino Dantas\inst{1} \and Deuslirio Junior\inst{2}}

\institute[UFG] % (optional)
{
  \inst{1}%
  Instituto de Informática\\
  Federal University of Goiás
  \and
  \inst{2}%
  Instituto de Informática\\
  Federal University of Goiás
}
\date{2019}
%The next statement creates the title page.

\frame[noframenumbering]{\titlepage}

%This page do not use the default titlepage
% \begin{frame}[noframenumbering]

% % \hspace{-0.08\textwidth}
% \LARGE{\centerline{{\usebeamercolor[fg]{title}Apresentações do Instituto de Informática}}}

% \vspace{0.1\textheight}
% \begin{columns}[]

% % \hspace{-0.2\textwidth}
% \column{0.40\textwidth}
%     \textbf{\small{\centerline{Altino Dantas}}}
%     \small{\centerline{ altinobasilio@inf.ufg.br}}
%     \small{\centerline{}}
    
% \column{0.40\textwidth}
%   \textbf{ \small{\centerline{Deuslirio Junior}}}
%     \small{\centerline{ deuslirio.junior@gmail.com}}
%     \small{\centerline{}}
% \column{0.20\textwidth}
% %   ALWAYS BLANK
% %   Sempre em branco, para não sobreescrever o local das logos

% \end{columns}
% \end{frame}
% \date[VLC 2014] % (optional)
% {Instituto de Informática, Outubro 2019}

%End of title page configuration block
%------------------------------------------------------------



% ------------------------------------------------







%---------------------------------------------------------
%This block of code is for the table of contents after
%the title page
\horiz
\begin{frame}
\frametitle{Table of Contents}
\tableofcontents
\end{frame}
%---------------------------------------------------------


\section{First section}

%---------------------------------------------------------
%Changing visivility of the text
\begin{frame}
\frametitle{Sample frame title}
This is a text in second frame. For the sake of showing an example.

\begin{itemize}
    \item<1-> Text visible on slide 1
    \item<2-> Text visible on slide 2
    \begin{itemize}
        \item test subitem
    \end{itemize}
    \item<3> Text visible on slides 3
    \item<4-> Text visible on slide 4
\end{itemize}
\end{frame}

 \begin{frame}{Problemas mais complexos}
\footnotesize
  \begin{ex}
Em uma versão da linguagem BASIC, o nome de uma variável é uma sequência de um ou dois caracteres alfanuméricos, em que letras maiúsculas e minúsculas não são distinguidas. Além disso, um nome de variável deve começar com uma letra e deve ser diferente das cinco sequências de dois caracteres reservadas para o uso de comandos. Quantos nomes diferentes de variáveis são possíveis nesta versão do BASIC?
  \end{ex}
%\pause

\begin{block}{Solução}
Considere $V$ o número de nomes possíveis de variáveis diferentes do BASIC. Seja $V_1$ a quantidade de varíaveis com um caractere e $V_2$ a quantidade de variáveis com dois caracteres. Pela regra da soma, $V=V_1+V_2$. Como as variáveis só podem começar com letras, temos que $V_1=26$. Pela regra do produto, há $26\cdot 36=936$ sequências de tamanho $2$ que comecem com uma letra e terminam com um caracter alfanumérico. Porém, não se deve usar $5$ variáveis reservadas. Assim, $V_2=26\cdot 36-5=931$. Logo, há $V=V_1+V_2 = 26+931=957$ nomes diferentes para variáveis nesta versão do BASIC.
  \end{block}

 \end{frame}
 
%---------------------------------------------------------
\verti
 \begin{frame}{Problemas mais complexos}
\footnotesize
  \begin{ex}
Em uma versão da linguagem BASIC, o nome de uma variável é uma sequência de um ou dois caracteres alfanuméricos, em que letras maiúsculas e minúsculas não são distinguidas. Além disso, um nome de variável deve começar com uma letra e deve ser diferente das cinco sequências de dois caracteres reservadas para o uso de comandos. Quantos nomes diferentes de variáveis são possíveis nesta versão do BASIC?
  \end{ex}
%\pause

\begin{block}{Solução}
Considere $V$ o número de nomes possíveis de variáveis diferentes do BASIC. Seja $V_1$ a quantidade de varíaveis com um caractere e $V_2$ a quantidade de variáveis com dois caracteres. Pela regra da soma, $V=V_1+V_2$. Como as variáveis só podem começar com letras, temos que $V_1=26$. Pela regra do produto, há $26\cdot 36=936$ sequências de tamanho $2$ que comecem com uma letra e terminam com um caracter alfanumérico. Porém, não se deve usar $5$ variáveis reservadas. Assim, $V_2=26\cdot 36-5=931$. Logo, há $V=V_1+V_2 = 26+931=957$ nomes diferentes para variáveis nesta versão do BASIC.
  \end{block}

 \end{frame}

%---------------------------------------------------------
%Example of the \pause command

\begin{frame}
\frametitle{Pause Example}
In this slide \pause

the text will be partially visible \pause

And finally everything will be there
\end{frame}

%---------------------------------------------------------

\section{Second section}

%---------------------------------------------------------
%Highlighting text
\begin{frame}
\frametitle{Sample frame title}

In this slide, some important text will be
\alert{highlighted} because it's important.
Please, don't abuse it.

\begin{block}{Remark}
Sample text
\end{block}

\begin{alertblock}{Important theorem}
Sample text in red box
\end{alertblock}

\begin{examples}
Sample text in green box. The title of the block is ``Examples".
\end{examples}
\end{frame}
%---------------------------------------------------------


%---------------------------------------------------------
%Two columns
\blank
\setslidebgcolor{Ocean}
\begin{frame}
\frametitle{Two-column slide}

\begin{columns}

\column{0.5\textwidth}
This is a text in first column.
$$E=mc^2$$
\begin{itemize}
\item First item
\item Second item
\end{itemize}

\column{0.5\textwidth}
This text will be in the second column
and on a second tought this is a nice looking
layout in some cases.
\begin{enumerate}
    \item First
    \item Second
\end{enumerate}
\end{columns}
\end{frame}
%---------------------------------------------------------


%---------------------------------------------------------
%Two columns
\mainpoint
\begin{frame}

    \begin{mainpointtext}
         Preliminary Empirical Study | Doc2vec
    \end{mainpointtext}


\end{frame}
%---------------------------------------------------------

% Example of changing background color 
\verti
\setslidebgcolor{Ocean}

\begin{frame}{Teste}
In this slide 
the text will be partially visible 
And finally everything will be there
\end{frame}

%---------------------------------------------------------


\end{document}